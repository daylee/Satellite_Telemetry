%%%%%%%%%%%%%%%%%%%%%%%%%%%%%%%%%%%%%%%%%%%%%%%%%%%%%%%%%%%%%%%%%%%%%%%%%%%%%%%%
%This section contains all of the appendices included in the manuscript.

\subsection{Orbital Simulation Data Values}

\begin{table}[!h]
\centering
\begin{tabular}{|p{0.4\textwidth}|p{0.2\textwidth}|}
\hline
\textbf{Parameter}         & \textbf{Value} \\ \hline
Radius of Earth (km)       & 6378.137       \\ \hline
Avg. Radius of Orbit (km)  & 7078.137       \\ \hline
Eccentricity               & 0              \\ \hline
Inclination (deg)          & 51.6           \\ \hline
RA of Ascending Node (deg) & 0              \\ \hline
Argument of Perigee (deg)  & 0              \\ \hline
True Anomaly (deg)         & 0              \\ \hline
GS Latitude (deg)          & 42.0236        \\ \hline
GS Longitude (deg)         & -93.6528       \\ \hline
\end{tabular}
\vspace{2 mm}
\caption{Parameters of the simulated CubeSat orbit and location of ground station.}
\label{OrbitalParameters}
\end{table}

\subsection{Antenna Design Parameters}

\begin{table}[!h]
\centering
\begin{tabular}{|p{0.4\textwidth}|p{0.2\textwidth}|}
\hline
\textbf{Parameter}          & \textbf{Value}                      \\ \hline
Length (m)                  & 0.0397                              \\ \hline
Width (m)                   & 0.0397                              \\ \hline
Strip Line Width (m)          & 0.0025                              \\ \hline
Notch Length (m)             & 0.0059                              \\ \hline
Notch Width (m)              & 0.0044                              \\ \hline
Patch Center {[}m,m,m{]}     & {[}0,0,0{]}                         \\ \hline
Feed Location {[}m,m,m{]}    & {[}-0.0441,0,0.0022{]}              \\ \hline
Height(m)                   & 0.0022                              \\ \hline
Ground Plane Length (m)       & 0.0882                              \\ \hline
Ground Plane Width (m)        & 0.0882                              \\ \hline
Patch Center Offset {[}m,m{]} & {[}0,0{]}                           \\ \hline
Feed Offset {[}m,m{]}        & {[}-0.0441,0{]}                     \\ \hline
Substrate                   & 1x1 dielectric {[}default values{]} \\ \hline
Substrate Thickness (m)                   & 0.00220435630882353 \\ \hline
Tilt                        & 0                                   \\ \hline
Tilt Axis                    & {[}1,0,0{]}                         \\ \hline
\end{tabular}
\vspace{2 mm}
\caption{Design Parameters of the Microstrip Patch Antenna used in developing the communications system model.}
\label{AntennaParameters}
\end{table}

\clearpage

\subsection{Downlink Constant Values}

\begin{table}[!ht]
\centering
\begin{tabular}{|p{0.4\textwidth}|p{0.2\textwidth}|}
\hline
\textbf{Parameter}      & \textbf{Value}            \\ \hline
frequency (Hz)          & 3.4*10\textasciicircum{}9 \\ \hline
Aperture Efficiency     & 0.0                       \\ \hline
TX Power (W)            & 10.0                      \\ \hline
TX Line Loss (dB)       & 0.0                       \\ \hline
Sky Temperature (K)      & 290                       \\ \hline
Polarization Loss (dB)   & 3.0                       \\ \hline
Pointing Error Loss (dB)  & 3.0                       \\ \hline
LNA Gain (dB)           & 16.0                      \\ \hline
LNA Noise Figure (dB)   & 1.5                       \\ \hline
RX Line Loss (dB)       & 3.0                       \\ \hline
Radio Noise Figure (dB) & 10.0                      \\ \hline
RX Antenna Gain (dB)    & 18.9                      \\ \hline
R (baud)                & 9600                      \\ \hline
Eb\_N0\_required (dB)   & 10                        \\ \hline
\end{tabular}
\vspace{2 mm}
\caption{Constant values used in the link budgeting step of the model.}
\label{DownlinkParameters}
\end{table}


\begin{comment}
\subsection{Sample Telemetry}

\begin{lstlisting}
{"_index":"Cube_beacon_01","_type":"_doc","_source":{"3.3V Bus Current":0.203,"3.3V Bus Voltage":3.299,"3.3V Input Current":0.105,"5V Bus Current":0.046,"5V Bus Voltage":5.00388,"5V Input Current":0.05,"@timestamp":"2020-01-16T20:46:20Z","ADC curr (?)":0.013250000000000001,"ADCS 3v3 current":-0.21858590000000003,"ADCS 3v3 voltage":3.288,"ADCS 5V current":-0.13066846000000001, "ADCS 5V voltage":4.9730099999999995,"ADCS Temp 0":22.942298435035127,"ADCS Temp 1":22.00006472534369,"ADCS VBatt current":-0.12939579999999998,"ADCS VBatt voltage":8.2126,"Battery Bus Current":0.067,"Battery Bus Voltage":8.2566,"Battery Current":0.16800000000000015,"Battery Temperature":14.91022252000002,"Battery Voltage":8.2698,"Channel 1 board temperature":20.37117822674668,"Channel 1 module temperature":20.28539843309295,"Channel 1 output current":0.004,"Channel 1 output voltage":0.0484,"Channel 1 panel current A":0.0008,"Channel 1 panel current B":0.0008,"Channel 1 panel voltage A":0.023960000000000002,"Channel 1 panel voltage B":0.023960000000000002,"Channel 2 board temperature":20.88575416931667,"Channel 2 module temperature":21.057240334344215,"Channel 2 output current":0.09,"Channel 2 output voltage":8.525,"Channel 2 panel current A":0.0158,"Channel 2 panel current B":0.0414,"Channel 2 panel voltage A":15.693800000000001,"Channel 2 panel voltage B":15.65786,"Channel 3 board temperature":21.742989427224302,"Channel 3 module temperature":22.085746726947946,"Channel 3 output current":0.161,"Channel 3 output voltage":8.551400000000001,"Channel 3 panel current A":0.1308,"Channel 3 panel current B":0.15200000000000002,"Channel 3 panel voltage A":12.189650000000002,"Channel 3 panel voltage B":1.60532,"Channel 4 board temperature":20.800003748429162,"Channel 4 module temperature":21.571581475708626,"Channel 4 output current":0.029,"Channel 4 output voltage":8.5008,"Channel 4 panel current A":0.022000000000000002,"Channel 4 panel current B":0.15500000000000003,"Channel 4 panel voltage A":15.6339,"Channel 4 panel voltage B":1.58735,"Channel 5 Module input voltage":5.106,"Channel 5 board temperature":24.140652432542993,"Channel 5 module temperature":24.48286419790361,"Channel 5 output current":0.007,"Channel 5 panel current A":0.0018,"Channel 5 panel current B":0.001,"Channel 6 Module input voltage":3.7,"Channel 6 board temperature":25.594515948494973,"Channel 6 module temperature":26.70534858484052,"Channel 6 output current":0.129,"Channel 6 panel current A":0.2884,"Channel 6 panel current B":0.013400000000000002,"Datamnt Usage":0.434,"EIMU current":0.00555,"EPS 3.3 Voltage":3.296,"EPS 3.3V Current":0.009800000000000001,"EPS 5V Current":0.0077,"EPS 5V Voltage":4.98477,"FCPU 3V3 Current":0.16869099999999998,"FCPU 3V3 Voltage":3.302,"FCPU Processor Temp":23.541594667502068,"FCPU Var 1":0,"FCPU Var 2":0,"FCPU Var 3":0,"FCPU Var 4":0,"FCPU Var 5":0,"FCPU Var 6":0,"Ioe States":237,"LI 3V3 Current":0.40367200000000003,"LI 3V3 Voltage":3.302,"LI VBATT Current":-0.19264799999999999,"LI VBATT Voltage":8.243400000000001,"Lithium #RX":0,"Lithium #TX":261648,"Lithium MSP430 Temp":0,"Lithium Op Count":11731,"Lithium PA Temp":22.85666520419977,"Lithium RSSI":97,"NumResets":548, "5VSbVolt":5.1200,"5VSbCurr":.0008,"3.3VSbVolt":3.3160,"3.3VSbCurr":.0005,"1.5VSbVolt"1.4930,"1.5VSbCurr":.0004,"Operation Mode":0,"Output Regulator Temperature":24.662146180000008,"PIM 3v3 current":-0.129,"PIM 3v3 voltage":0,"PIM VBatt monitor":0,"RTC Unix Time":1579207577,"SD Usage":0,"SD V (?)":0,"STAMP GPIO States":29809,"TBEx Payload Current":-0.227,"TBEx Payload Voltage":0,"avgNumActiveTasks1":1.64,"avgNumActiveTasks15":1.81,"avgNumActiveTasks5":1.82,"data_checksum_passed":true,"flag":3,"freeMemPlusCache":107632,"header_checksum_passed":true,"pid":0,"sid":83,"telemetry_archive":false,"telemetry_archive_file_num":0,"totNumProcesses":41,"usedMemMinusCache":19272}}
\end{lstlisting}

\end{comment}

\subsection{Sample Specifications for the Modeled CubeSat Communications System}

This appendix consists of 12 separate specifications that were developed for this case study. Each specification, written in MLTL, mathematically describes the expected behavior of system variables while removing potential risk of misinterpretation.

\subsubsection{\textbf{Necessity of Line-of-Sight Communication}}
“Communication between the spacecraft and ground station can only occur when the spacecraft is above the horizon, ie. the elevation angle of the spacecraft must be above 0. (Line of sight conditions are met.)"
\[ Atomic\:Propositions \begin{cases}
  \varphi_1 & (El_{SAT} > 0) \\
  \varphi_2 & COMM == 1) \\
\end{cases} \]
\begin{equation}
    \label{Spec 1}
    G_{[0,M]} \{(\varphi_2 \rightarrow \varphi_1)\}
\end{equation} 
This specification stems from common sense, as there must be line of sight between the CubeSat and the ground station.

\subsubsection{\textbf{Azimuth Angle Bounds}}
“The azimuth angle of the CubeSat, Az$_{SAT}$, will be bounded by 0 and 360 at all instances. The current value of Az$_{SAT}$ will not vary more than 1$^{\circ}$ from the previous time step."
\[ Atomic\:Propositions \begin{cases}
  \varphi_1 & (Az_{SAT} \geq -180) \\
  \varphi_2 & (Az_{SAT} \leq 180) \\
  \psi & (abs(Az_{SAT} - Az_{SAT(i-1)}) < 1.0) \\
\end{cases} \]
\begin{equation}
    \label{Spec 2}
    G_{[0,M]} \{(\varphi_1 \wedge \varphi_2 \wedge \psi)\}
\end{equation} 

\subsubsection{\textbf{Elevation Angle Bounds}}
“The elevation angle of the CubeSat, El$_{SAT}$, will be bounded by -90 and 90 at all instances."
\[ Atomic\:Propositions \begin{cases}
  \varphi_1 & (El_{SAT} \geq -90) \\
  \varphi_2 & (El_{SAT} \leq 90) \\
  \psi & (abs(El_{SAT} - El_{SAT(i-1)}) < 1.0) \\
\end{cases} \]
\begin{equation}
    \label{Spec 3}
    G_{[0,M]} \{(\varphi_1 \wedge \varphi_2 \wedge \psi)\}
\end{equation} 

\subsubsection{\textbf{Orbit Number Check}}
“The orbit number of the spacecraft provided by the model will never be less than 0. The orbit number of the current time step will be equal to or greater than the orbit number from the previous time step."
\[ Atomic\:Propositions \begin{cases}
  \varphi_1 & (OrbNum \geq 0) \\
  \psi_1 & (OrbNum \geq OrbNum_{i-1})\\
\end{cases} \]
\begin{equation}
    \label{Spec 4}
    G_{[0,M]} \{(\varphi_1 \wedge \psi_1)\}
\end{equation} 
This information is crucial to knowing how many passes the CubeSat has made, and also acts as a check to ensure the information pulled from the model is sensible.

\subsubsection{\textbf{Radio Temperature Variation}}
“The temperature of the radio, Radio\_PA\_Temp, will not vary more 1$^{\circ}$ from the previous time step."
\[ Atomic\:Propositions \begin{cases}
  \varphi & (abs(Radio\_Temp - Radio\_Temp_{i-1} \leq 1.0)) \\
\end{cases} \]
\begin{equation}
    \label{Spec 5}
    G_{[0,M]} \{(\varphi)\}
\end{equation} 

\subsubsection{\textbf{Radio Power Amplifier Temperature Variation}}
“The temperature of the radio's power amplifier, Radio\_PA\_Temp, will not vary more 1$^{\circ}$ from the previous time step."
\[ Atomic\:Propositions \begin{cases}
  \varphi & (abs(Radio\_PA\_Temp - Radio\_PA\_Temp_{i-1} \leq 1.0)) \\
\end{cases} \]
\begin{equation}
    \label{Spec 6}
    G_{[0,M]} \{(\varphi)\}
\end{equation} 

\subsubsection{\textbf{Radio 3.3V Voltage Variation}}
“The Radio 3.3V line, 3.3VRadioVolt, must not vary more than 10\% from the desired 3.3V value. It must also not vary more than .5V from the previous time step."
\[ Atomic\:Propositions \begin{cases}
  \varphi_1 & (3.3VRadioVolt \geq 2.97V) \\
  \varphi_2 & (3.3VRadioVolt \leq 3.63V) \\
  \psi & (abs(3.3VRadioVolt - 3.3VRadioVolt_{i-1}) < .5)\\
\end{cases} \]
\begin{equation}
    \label{Spec 7}
    G_{[0,M]} \{(\varphi_1 \wedge \varphi_2 \wedge \psi)\}
\end{equation} 

\subsubsection{\textbf{Radio 3.3V Current Variation}}
“The Radio 3.3V line must not have a current, 3.3VRadioCurr, varying more than 5\% from
the desired current value, 3.3VRadioExpCurr. It must also not vary more than .05A from the value of the previous time step."
\[ Atomic\:Propositions \begin{cases}
  \varphi_1 & (3.3VRadioCurr \geq .95*3.3VRadioExpCurr) \\
  \varphi_2 & (3.3VRadioCurr \leq 1.05*3.3VRadioExpCurr) \\
  \psi & (abs(3.3VRadioCurr - 3.3VRadioCurr_{i-1}) \leq .05)\\
\end{cases} \]
\begin{equation}
    \label{Spec 8}
    G_{[0,M]} \{(\varphi_1 \wedge \varphi_2 \wedge \psi)\}
\end{equation} 

\subsubsection{\textbf{Radio VBatt Voltage Variation}}
“The Radio VBatt voltage, RadioVBattVolt, must not vary more than 10\% from the desired 8.3V value. It must also not vary more than .5V from the previous time step."
\[ Atomic\:Propositions \begin{cases}
  \varphi_1 & (RadioVBattVolt \geq 7.47V) \\
  \varphi_2 & (RadioVBattVolt \leq 9.13V) \\
  \psi & (abs(RadioVBattVolt - RadioVBattVolt_{i-1}) < .5)\\
\end{cases} \]
\begin{equation}
    \label{Spec 9}
    G_{[0,M]} \{(\varphi_1 \wedge \varphi_2 \wedge \psi)\}
\end{equation} 

\subsubsection{\textbf{Radio VBatt Current Variation}}
“The Radio VBatt voltage line must not have a current, RadioVBattCurr, varying more than 5\% from the desired current value, RadioVBattCurrExp. It must also not vary more than .05A from the value of the previous time step."
\[ Atomic\:Propositions \begin{cases}
  \varphi_1 & (RadioVBattCurr \geq .95*RadioVBattCurrExp) \\
  \varphi_2 & (RadioVBattCurr \leq 1.05*RadioVBattCurrExp) \\
  \psi & (abs(RadioVBattCurr - RadioVBattCurr_{i-1}) \leq .05)\\
\end{cases} \]
\begin{equation}
    \label{Spec 10}
    G_{[0,M]} \{(\varphi_1 \wedge \varphi_2 \wedge \psi)\}
\end{equation} 

\subsubsection{\textbf{Transmit Count Check}}
“The number of times the CubeSat transmits to the ground station, Radio\_TX, will be greater than the previous time the transmission was received."
\[ Atomic\:Propositions \begin{cases}
  \varphi & (Radio\_TX > Radio\_TX_{i-1})\\
\end{cases} \]
\begin{equation}
    \label{Spec 11}
    G_{[0,M]} \{(\varphi)\}
\end{equation} 

The number won't necessarily be greater by exactly 1, since the ground station could conceivably not receive a transmission.

\subsubsection{\textbf{Operation Count Check}}
“The number of times the CubeSat performs operations, Radio\_Op\_Count, will be greater than the previous time the transmission was received."
\[ Atomic\:Propositions \begin{cases}
  \varphi & (Radio\_Op\_Count > Radio\_Op\_Count_{i-1})\\
\end{cases} \]
\begin{equation}
    \label{Spec 12}
    G_{[0,M]} \{(\varphi)\}
\end{equation} 

The number won't necessarily be greater by exactly 1, since the ground station could conceivably not receive a transmission.

\begin{comment}
\subsubsection{\textbf{Spacecraft Attitude Bounds}}
“The orientation angles X,Y,Z measured by the spacecraft sensors shall be bounded between 0 and 360 degrees in all instances."
\[ Atomic\:Propositions \begin{cases}
  \varphi_1 & (X $\geq$ 0) \\
  \varphi_2 & (Y $\geq$ 0) \\
  \varphi_2 & (Z $\geq$ 0) \\
  \psi_1 & (X $\leq$ 360) \\
  \psi_2 & (Y $\leq$ 360) \\
  \psi_2 & (Z $\leq$ 360) \\
\end{cases} \]
\begin{equation}
    \label{Spec 13}
    G_{[0,M]} \{(\varphi_1 \wedge \psi_1) \wedge (\varphi_2 \wedge \psi_2) \wedge (\varphi_3 \wedge \psi_3)\}
\end{equation} 
\end{comment}